\documentclass{article}
\begin{document}
\title{A RESEARCH REPORT ON RAMPANT THEFT AROUND MAKERERE UNIVERSITY}
\author{SSEKIRANDA BRIAN }
\maketitle
\section{Introduction}
Theft is an action of stealing someone’s property, endangering someone’s life or even killing him/her. The intention of this report is to highlight the cause and what may be the solutions to curb down this rampant threat!
This report also mainly focuses on the some tactics used by the thugs to harm people or even killing them and taking their property. It is also based on measures security agencies have tried to lay down to this problem.
Methods:
This report was conducted by interviewing several mass of people on individual basis. About 1000 citizens were interviewed to give the cause and how he or she can solve that problem (solution).

\section{Causes}
\begin{enumerate}
\item Rampant use of drugs among the youth around Kikoni , Nankulabye , kikumikikumi , Bwaise and those suburb areas neighboring the University.
\item High growth rate of unemployment in city which has forced the youth to run away from the city and hide in these cheaper areas. This has forced them to still since they are doing nothing at all.
\item Target market is also the cause where by every year the campus has to welcome freshers. So these thieves use this chance to do their Job
\item Weakness in security agencies where by the personnel employed is inadequate to survey all the places around the campus.
\item Corruption within the society of people of Makerere has also accelerated the existence of thieves around Makerere
\end{enumerate}
\section{Solutions}
\begin{enumerate}
\item Health education and sensitization in youth should be done in order to show them the dangerous that results from taking and misuse of drugs like Kurba , Marijuana , petrol and others. 
\item	Makerere University should request for more security  personnels  to tighten their security more especially during the incoming of freshers
\item Community policing should be emphasized among university students since they are the most victicms
\item Harsh policies that lead people to run away from city like chasing away hawkers. This increases more people in places around like Maker ere causing unemployment which leads people to still those who are working.
\end{enumerate}
\section{Conclusion}
Makerere University with corporation from the government should work tooth and nail to curb down this situation by sensitizing people on drug abuse, designing curriculum that gives how to create jobs other than how to look for jobs, community policing among the students at Makerere University should be emphasized and condemning corruption.
\end{document}